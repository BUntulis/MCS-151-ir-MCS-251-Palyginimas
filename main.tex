\documentclass{article}
\usepackage{graphicx} % Required for inserting images

\title{KA - Architektūrų palyginimas MCS-151 tarp MCS-251}
\author{Benas Untulis}
\date{2024-12-18}

\begin{document}

\maketitle

\section{Elementinė kompiuterio/procesoriaus bazė}
\chapter{\normalfont{
\textbf{MCS-151} procesorius buvo pagamintas naudojant labai didelio integracijos masto (VLSI) integrinius grandynus, kurie leido sumažinti dydį ir energijos suvartojimą. Fizinis įrenginys buvo kompaktiškas ir lengvas, palyginti su senesnėmis architektūromis, sunaudodamas minimalų energijos kiekį (tipiškai apie 1 W).\newline\newline
\textbf{MCS-251} buvo patobulintas MCS-151 variantas, sukurtas naudojant modernias integrinio grandyno technologijas (VLSI arba monokristalinius procesorius). Jo pagrindinė bazė suteikė geresnį efektyvumą, didesnį greitį ir mažesnį energijos suvartojimą. Energijos sunaudojimas buvo dar efektyvesnis už pirmtaką.
}
 
\section{Architektūros tipas}
\chapter{\normalfont{
\textbf{MCS-151} buvo registrinės architektūros procesorius su pagrindiniais registrų rinkiniais, leidžiančiais efektyviai manipuliuoti duomenimis. Architektūra taip pat palaikė akumuliatoriaus (accumulator-based) dizainą, naudojamą pagrindinei aritmetikai ir duomenų judėjimui.\newline\newline
\textbf{MCS-251} buvo registrinė architektūra, bet turėjo platesnį komandų rinkinį ir geriau išnaudojamą atmintį, lyginant su pirmtaku. Patobulinti registrai leido lengviau kurti sudėtingesnes programas.
}

\section{Adresavimo schemos}
\chapter{\normalfont{
\textbf{MCS-151} buvo vieno adreso mašina, kurioje dauguma komandų naudojo akumuliatorių kaip pagrindinį operandą.\newline\newline
\textbf{MCS-251} palaikė dviejų adresų modelį, kuris leido efektyviau valdyti duomenų judėjimą ir sumažino komandų vykdymo ciklų skaičių.
}

\section{Registrai}
\chapter{\normalfont{

}
 
\section{Požymių bitai}
\chapter{\normalfont{\textbf{Atsakymas:}\newline 

}
 
\section{Duomenų plotis}
\chapter{\normalfont{

}
 
\section{Atminties struktūra}
\chapter{\normalfont{\textbf{Atsakymas:}\newline 

}
 
\section{Virtualioji atmintis}
\chapter{\normalfont{

}
 
\section{Komandų sistema}
\chapter{\normalfont{

}
 
\section{Adresavimo būdai}
\chapter{\normalfont{

}
 
\section{Įvesties-išvesties galimybės}
\chapter{\normalfont{

}
 
\section{Pertrauktys}
\chapter{\normalfont{

}
 
\section{Duomenų tipai}
\chapter{\normalfont{

}
 
\section{Greitaveika}
\chapter{\normalfont{

}
 
\section{Spartinančioji atmintis}
\chapter{\normalfont{

}
 
\section{Tipinės taikymo sritys}
\chapter{\normalfont{

}
 
\section{Programinė įranga}
\chapter{\normalfont{

}
 
\section{Emuliatoriai}
\chapter{\normalfont{

}
 


\section{Naudoti šaltiniai}
\chapter{\normalfont{
\begin{itemize}
    \item Wikipedia. (2024). Intel 8051. Prieiga per: https://en.wikipedia.org/wiki/Intel_8051
\end{itemize}
}
 
\end{document}
